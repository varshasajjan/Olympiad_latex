\documentclass[12pt,a4paper]{article}
\usepackage{amsmath,amssymb}
\usepackage{geometry}
\geometry{margin=1in}
\usepackage{graphicx}
\usepackage{titlesec}
\titleformat{\section}{\normalfont\Large\bfseries}{1962/\thesection.}{1em}{}


\begin{document}
\begin{center}
\begin{minipage}{0.2\textwidth}
    \includegraphics[width=\linewidth]{iiit_logo.png} 
\end{minipage}
\hfill
\begin{minipage}{0.75\textwidth}
    \centering
    {\Large \textbf{Fourth International Olympiad, 1962}}\\[1ex]
    \textbf{Name: Varshini G N} \\[0.5ex]
    ID:COMETFWC031\\
    \textbf{Date:} \today
\end{minipage}
\end{center}

\vspace{1em}
\title{Fourth International Olympiad, 1962}
\hrule
\vspace{1em}

\section{}
Find the smallest natural number $n$ which has the following properties:
\begin{itemize}
    \item[(a)] Its decimal representation has 6 as the last digit.
    \item[(b)] If the last digit 6 is erased and placed in front of the remaining digits, the resulting number is four times as large as the original number $n$.
\end{itemize}

\section{}
Determine all real numbers $x$ which satisfy the inequality:
\[
\sqrt{3 - x} - \sqrt{x + 1} > \frac{1}{2}.
\]

\section{}
Consider the cube $ABCDA'A''B''C''D''$ (where $ABCD$ and $A'B'C'D'$ are the upper and lower bases, and edges $AA'$, $BB'$, $CC'$, $DD'$ are vertical).  
The point $X$ moves at constant speed along the perimeter of square $ABCD$ in the direction $A \rightarrow B \rightarrow C \rightarrow D \rightarrow A$,  
and the point $Y$ moves at the same speed along the perimeter of square $B'C'CBB'$ in the direction $B' \rightarrow C' \rightarrow C \rightarrow B \rightarrow B'$.  
Points $X$ and $Y$ start simultaneously from positions $A$ and $B'$, respectively.

Determine and draw the locus of the midpoints of the segments $XY$.

\section{}
Solve the equation:
\[
\cos^2 x + \cos^2 2x + \cos^2 3x = 1.
\]

\section{}
On the circle $K$, three distinct points $A$, $B$, and $C$ are given.  
Construct (using only straightedge and compasses) a fourth point $D$ on $K$ such that a circle can be inscribed in the quadrilateral $ABCD$.

\section{}
Consider an isosceles triangle. Let $r$ be the radius of its circumscribed circle and $\rho$ the radius of its inscribed circle.  
Prove that the distance $d$ between the centers of these two circles is:
\[
d = \sqrt{r(r - 2\rho)}.
\]

\section{}
The tetrahedron $SABC$ has the following property:  
there exist five spheres, each tangent to the edges $SA$, $SB$, $SC$, $BC$, $CA$, $AB$, or to their extensions.

\begin{itemize}
    \item[(a)] Prove that the tetrahedron $SABC$ is regular.
    \item[(b)] Prove conversely that for every regular tetrahedron, five such spheres exist.
\end{itemize}

\end{document}
